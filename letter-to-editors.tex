% LaTeX rebuttal letter example. 
% 
% Copyright 2019 Friedemann Zenke, zenkelab.org
%
% Based on examples by Dirk Eddelbuettel, Fran and others from 
% https://tex.stackexchange.com/questions/2317/latex-style-or-macro-for-detailed-response-to-referee-report
% 
% Licensed under cc by-sa 3.0 with attribution required.

\documentclass[11pt]{article}
\usepackage[utf8]{inputenc}
\usepackage{fullpage}

% import Eq and Section references from the main manuscript where needed
% \usepackage{xr}
% \externaldocument{manuscript}

% package needed for optional arguments
\usepackage{xifthen}
% define counters for reviewers and their points
\newcounter{reviewer}
\setcounter{reviewer}{0}
\newcounter{point}[reviewer]
\setcounter{point}{0}

% This refines the format of how the reviewer/point reference will appear.
\renewcommand{\thepoint}{P\,\thereviewer.\arabic{point}} 

% command declarations for reviewer points and our responses
\newcommand{\reviewersection}{\stepcounter{reviewer} \bigskip \hrule
                  \section*{Reviewer \thereviewer}}

\newenvironment{point}
   {\refstepcounter{point} \bigskip \noindent {\textbf{Reviewer~Point~\thepoint} } ---\ }
   {\par }

\newcommand{\shortpoint}[1]{\refstepcounter{point}  \bigskip \noindent 
	{\textbf{Reviewer~Point~\thepoint} } ---~#1\par }

\newenvironment{reply}
   {\medskip \noindent \begin{sf}\textbf{Reply}:\  }
   {\medskip \end{sf}}

\newcommand{\shortreply}[2][]{\medskip \noindent \begin{sf}\textbf{Reply}:\  #2
	\ifthenelse{\equal{#1}{}}{}{ \hfill \footnotesize (#1)}%
	\medskip \end{sf}}

\begin{document}

Dear sirs:

    Please find attached the first revision of the paper {\em Intra-family links
      in the analysis of marital networks} for your consideration. Following
    your and the reviewers' advice, the paper has improved greatly in the
    content as well as in the presentation areas. These are the general changes
    that the paper has undergone:\begin{itemize}
    \item The paper has now 26 pages, as opposed to the original 20; this
      accounts for the inclusion of new illustrations and a totally new social
      network analyzed.
    \item There are now 33 references, as opposed to the original 28; there are
      9 new references, since 4 original references that were less germane to
      the topic of the paper has been dropped.
    \item A new section that describes and compares the high-level features of
      the two datasets under study has been added.
    \item Illustrations for all methods tested have been added.
    \item Explanations of the different methodologies studied have been
      clarified and expanded where needed.
    \end{itemize}

Please find below a point-by-point response to all comments by the reviewers.

\section*{Response to the reviewers}

We would like to thank the editor and reviewers for the attention that our work has received. We will address their concerns next

\reviewersection

\begin{point}
The article delves into the often overlooked dimension of matrimonial networks:
intra-family marriages. Specifically, the authors explore three methodologies
for incorporating intra-family ties into marital networks. They endorse a
procedure that duplicates nodes for families with intra-family marriages, adding
new edges to account for these ties. Despite this method being highlighted for
its computational simplicity and conceptual soundness, there are some
recommendations and suggestions to improve the quality of the paper:
	\label{pt:1:1}
      \end{point}

      \begin{reply}
        This is an accurate summary of our work. We thank the reviewer for
        his/her recommendations, and will address every point below one by one.
        \end{reply}

\begin{point}
The structure of the article does not represent the contribution of the article
to scientific research. Authors should clearly explain what their contributions
are, specifying the main field of research concerned;
\end{point}

    \begin{reply}
     The contribution of the article to scientific research
     is the proposal of a simple procedure, called {\em duplicated nodes}, that
     incorporates self-loops into graphs by duplicating nodes of families with
     self-loops and linking them with an edge whose weight is equivalent to the
     number of self-marriages. This procedure allows centrality measures such as PageRank
     and betweenness centrality to numerically include these self-loops into their
     results. This has been explained more clearly in the abstract and the
     introduction. We have also added the analysis of a different network,
     applying the method proposed, and finding how incorporating self-loops yield
     a better representation of status via centrality measures.
     \end{reply}


\begin{point}
    While the paper discusses matrimonial connections in a specific historical
    and geographical milieu, it might not have elaborated enough on the
    historical, cultural, and social factors that drive intra-family marriages
    in different societies.
\end{point}

    \begin{reply}
      We introduced references to Luke et al.~(2004) and Kuper (2001) to
       highlight the cultural differences between East Asia and Africa and
       between religions in the social acceptance of intra-family marriages. To
       go further might lead us astray, since the paper proposes a computational
       method. However,  the new method proposed will help to better study the historical, cultural and social factors that drive
       self-marriages in different societies.
    \end{reply}

\begin{point}
  The study seems to focus on a specific kind of network. While the findings
  might be significant within this context, their applicability to other marital
  networks from different cultures or historical periods remains uncertain.
\end{point}

    \begin{reply}
         The method is now applied to a second marriage networks of 1271
         Taiwanese elite families from 1895 to 1996. This second network is
         quite different from the Venice network, as shown in the comparison in
         (the totally new) Section 3: It is more sparse, covers a shorter period
         of time, and has fewer intra-family marriages.  We believe that
         applying our method to two marriage networks with substantial
         differences provides a better test of its robustness.
    \end{reply}

\begin{point}
    By duplicating nodes for families with intra-family marriages, there's a risk
    of oversimplifying the network and possibly overlooking some nuanced
    connections or relationships. How did the authors overcome these issues?
\end{point}

\begin{reply}
    This is a very good question indeed. Short answer: by ensuring that the new,
    duplicated, nodes, are structurally equivalent to the ones they replicate,
    we avoid this risk of simplification; as a matter of fact, we go in the
    opposite direction: dropping self-loops {\em does} simplify the network to
    the point that some measures might not reflect correctly the status of the
    family.

    There is a long answer too. The {\sf duplicated mode} procedure, introduced
    in this paper, turns pseudographs into unipartite or regular graphs, by
    converting a type of multi-edge, self-loops, into an additional loop and
    edge; as opposed to the standard way, which simply drops the self-loops. So
    it simplifies the model of the graph (going from a complex, multi-edge)
    model, to a simple, regular, single-edge/weight model. However, by
    simplifying the model, we win the capability to perform analysis using
    off-the-shelf tools without losing a iota of interpretability or adding
    ad-hoc artifacts. As a matter of fact, multigraphs incorporate all kind of
    relationship, modeling them as different kind of edges: political ties might
    use a different type of edge than commercial or familiar ties. However,
    usual social network analysis will simply coalesce these ties into a single
    edge, losing the possible differences and nuances between those
    relationships. Dealing with these kind of relationships in this paper is
    probably out of scope; however, the procedure of duplication used in this
    paper paves the way for introducing other {\em regularization} procedures by
    mapping these nuanced relationships to a {\em regular} graph in such a way
    that the nuances in the relationship are not lost, but at the same time you
    can compute and compare with prior work that uses regular centrality
    measures. So if we would have to generalize our methodology, regularization
    of multigraphs should be done in such a way that the structural properties
    of the nodes in the network are conserved, while at the same time the
    nuanced ties can numerically be incorporated into the regular graph.

    In the paper, the fact that this regularization of the network does not
    simplify it is validated by analyzing the resulting values of the centrality
    measures with the {\sf duplication} method in both networks tested (Venice
    and Taiwan) and comparing them with the status of the families involved as
    reflected in the original paper that published the datasets.
\end{reply}

\begin{point}
    Despite the English writing is quite good, the methods and architecture
    proposed are not convincing and very confusing. Please improve it and
    provide better motivation for the chosen approach, and, in the methods
    section, communicate in a simple way to make the article understandable even
    those who do not know this area.
  \end{point}

  \begin{reply}
    We have added illustrations for all the methods proposed, so that it is
    clear what happens with the involved nodes, and their connections, with
    every one of them; these are now Figures 2, 3 and 5. The explanation of
    every one of the methods has been revised and improved, referring to these
    illustrations.

    The motivation for the chosen approach has been expanded in the
    Introduction: essentially, social network analysis of marriage networks drop
    a very important part of it: intra-family ties. Our method is able to
    incorporate it into the analysis, and in a way that leverages existing
    measures and software infrastructure.
  \end{reply}

\begin{point}
    The paragraph structure, information flow and connectivity of the article can be improved.
\end{point}

\begin{reply}
  We thoroughly reviewed the architecture of the paper, by rewriting the
  abstract and the introduction but also changing the paragraph structure in
  order to make the paper easier to read. We have now divided Section 4 (which
  was previously section 3) in several subsections to highlight every different
  method tested (and how the finally selected method, {\sf duplicated nodes}, is
  able to reflect better status in centrality measures for the Taiwanese network.
\end{reply}

% Begin a new reviewer section
\reviewersection

\begin{point}
  The authors proposed three different ways of incorporating loops into marital
  network analysis. At first, a selection of methods to convert bipartite
  networks into single-mode networks is proposed. These methods are then applied
  to a dataset of marriages in the Republic of Venice. Finally, the best method
  is selected. Although the paper topic is significant, I found it unclear and
  difficult to read. These are my comments to improve it:
\end{point}

\begin{reply}
  Thanks for this summary, which accurately reflects our intent with this
  paper. We have tried to make it clearer and easier to read in this new
  version, that incorporates new analysis, as well as improvements to the
  structure and individual claims. We will now address the rest of your
  concerns.
\end{reply}


\begin{point}
    The number of references in Section 2 (State of the art) are not
    sufficient. The authors can make a deeper study on the literature of the
    more recent years.
\end{point}

\begin{reply}
  Part of this Section has been moved to the Introduction; this Section is
  called now ``A survey of marriage networks'' and has a narrower focus on
  social network analysis applied to marriage networks. It is divided by
  different geographical milieus and extended with added Zamborain-Mason (2017)
  (in the introduction), Tacket (2020), Cruz et al. (2017), Haim et al. (2021) and
  Naidu et al. (2021). We have also added new references to papers that refer
  explicitly to self-loops: Smith et al., 2021 and Lichoti et al., 2016.

  Unfortunately, the lack of good tools to take into account self-loops in
  marital networks has probably contributed to the dearth of literature in this
  area; we expect that this paper will help researchers bridge the existing gap.
\end{reply}


\begin{point}
    Section 3 (Methods) is very confusing. In my opinion, a mathematical
    formulation of the problem is missing. Definitions of graph, loops, and so
    on should be inserted together with a list of steps or even a pseudocode
    with clearly states the procedure used by the authors to incorporated
    loops. These contents can become a subsection. Moreover, an other subsection
    to clearly explain the dataset could be added.
\end{point}

\begin{reply}
Every method now contains illustrations of what is meant by every term,
illustrating also the concept of self-loop (intra-marriages in our case). A new
section, Section 3, is now devoted to explain the datasets used, the Venetian
marriage network as well as the new one analyzed in this paper, the Taiwanese
elite family network. A more precise definition of the graph that is generally
used to model a social network is now in the Introduction. Please note that we
are not using specific models in this paper, instead using the regular graph
representation of marital networks where nodes are families and edges represent
matrimonial links between them.
\end{reply}


\begin{point}
  Section 4 (Results) should contain a list of items of the three used methods
  and also contain plots to show the experiment results. Moreover, only one
  network with 348 nodes and 11842 edges has been used in the experiments. I am
  not sure it is sufficient to support the obtained results.
\end{point}

\begin{reply}
   Former Sections 3 and 4 have been merged into Section 4, which is now divided
   in several subsections, every one for one of the proposed
   methods. Additionally, Section 3 now contains a summary of the two datasets
   used. The second dataset of Taiwanese elite families is analyzed in
   Subsection 4.4. Since this dataset is totally different in most high-level
   aspects (such as transitivity, average degree and even total number of
   self-loops) and we still obtain significant results (validated with the
   analysis made in the paper that published the results in the first place),
   this implies that our proposed {\sf duplicated nodes} method is robust enough
   to be applied to a wide variety of networks with self-loops, even beyond the
   context, marital networks, that is the focus of this paper.
\end{reply} 

\begin{point}
  The bibliography part requires an accurate revision.
\end{point}

\begin{reply}
  The bibliography has been revised and new references have been added, both to
  the literature on marriage networks and to the literature on social networks
  analysis and self-loops. To avoid an unnecessarily long list of references, we
  have also removed some works that dealt with marriage networks but did not
  make specific use of social network analysis. Minor errors have also been
  fixed, specifically errors in the paper on Ragusan families, where the
  proceedings editor had been wrongly listed as the paper author.
\end{reply} 

\begin{point}
    Links along the paper, bibliography and declarations also require revision.
\end{point}

\begin{reply}

  Every part of the paper has been revised thoroughly. We thank the reviewer for
  these helpful suggestions.

\end{reply}

\end{document}